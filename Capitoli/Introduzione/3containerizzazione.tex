%!TEX root = ../../00main.tex

\section{Containerizzazione}

\subsection{Panoramica}
Con il termine containerizzazione si intende una para-virtualizzazione dell'ambiente applicativo che consente di eseguire software, librerie, dipendenze e tutte le componenti necessarie, in un processo isolato che prende il nome di contenitore. Esso può essere considerato un vero e proprio eseguibile che non dipende da alcuna sorgente esterna. Questo lo rende estremamente portatile e affidabile in quanto può essere eseguito e trasferito in ogni tipo di ambiente e d'infrastruttura.
L'idea alla base del paradigma non è in realtà nuova in quanto sfrutta una funzionalità del kernel Linux presente già dalla versione 2.6.24 che permette l'isolamento e la gestione delle risorse di uno o più processi.

\subsection{Vantaggi}
L'utilizzo di questa tecnologia ha rivoluzionato il modus operandi del settore IT, dello sviluppo software e della loro distribuzione su infrastrutture Cloud. Questo grazie a tre principali vantaggi:
\begin{itemize}
\item \textbf{Portabilità} - Date le sue caratteristiche, un'applicazione containerizzata è per natura adattabile ed eseguibile in ogni tipo di ambiente. 
\item \textbf{Efficienza} - I contenitori hanno la capacità di condividere il kernel della macchina ospitante evitando così di dover disporre di risorse hardware e software dedicate.
\item \textbf{Ottimizzazione dello spazio} - L'immagine dell'applicazione che viene eseguita all'interno del contenitore, incapsula solo ed unicamente le componenti e le informazioni necessarie alla sua esecuzione.
\item \textbf{Ambienti di sviluppo innovativi} - Una nuova strategia di sviluppo che si sta affermando negli ultimi anni è quella del DevOps e la containerizzazione è un punto cardine di questo nuovo paradigma. Per la natura stessa dei contenitori, gli sviluppatori possono condividere facilmente il loro software, eliminando così problemi di compatibilità dettati da ambienti di sviluppo e di esecuzione differenti. 
\end{itemize}

\subsection{Macchina Virtuale e Containerizzazione}
Una macchina virtuale è una rappresentazione virtuale delle risorse hardware e software di un sistema informatico. Essa ha il compito dunque di ricreare CPU, memoria, interfaccia di rete e storage e di eseguire un intero sistema operativo.
Un contenitore si basa sul kernel del sistema operativo host e contiene solo le componenti necessarie all'esecuzione dell'applicativo containerizzato.
La somiglianza tra containerizzazione e virtualizzazione sta nel fatto che entrambe consentono l'isolamento completo delle applicazioni, che le rende operative su più ambienti. La differenza sta invece nelle dimensioni, nella portabilità e nell'efficienza.