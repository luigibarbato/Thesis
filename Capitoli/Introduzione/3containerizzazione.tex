%!TEX root = ../../00main.tex

\section{Containerizzazione}
La virtualizzazione è una tecnologia che permette di mettere a disposizione risorse hardware come CPU e memoria sottoforma di risorse virtuali. Attraverso la virtualizzazione è possibile utilizzare delle risorse virtuali allo stesso modo di come si farebbe con delle risorse fisiche permettendo di migliorare la scalabilità e i carichi di lavoro, usando al contempo un minor numero di macchine, una quantità di energia elettrica ridotta, generando così un risparmio sui costi di infrastruttura e gestione.
Uno dei principali vantaggi della virtualizzazione è la razionalizzazione e l'ottimizzazione delle risorse hardware in quanto più macchine virtuali possono girare contemporaneamente su un sistema fisico condividendo le risorse della piattaforma.
La metodologia più semplice di virtualizzazione prevede la creazione di una macchina virtuale che va a simulare un'intero sistema fisico con risorse come CPU, memoria e componenti di reti. Una macchina virtuale dovrà quindi essere utilizzata come un normale computer desktop, eseguendo un proprio sistema operativo e rimanendo completamente isolata dall'hardware fisico sottostante.
Con il termine containerizzazione si intende una para-virtualizzazione dell'ambiente applicativo che consente di eseguire software, librerie, dipendenze e tutte le componenti necessarie, in un processo isolato che prende il nome di contenitore. Esso può essere considerato un vero e proprio eseguibile che non dipende da alcuna sorgente esterna. Questo lo rende estremamente portatile e affidabile in quanto può essere eseguito e trasferito in ogni tipo di ambiente e d'infrastruttura.
L'idea alla base del paradigma non è in realtà nuova in quanto sfrutta una funzionalità del kernel Linux presente già dalla versione 2.6.24 che permette l'isolamento e la gestione delle risorse di uno o più processi.

\subsection{Vantaggi}
Il grande successo della virtualizzazione basata su container deriva dai grandi vantaggi che essa garantisce. Tra essi troviamo:
\begin{itemize}
\item \textbf{Portabilità} - I container sono altamente portabili in quanto contengono al loro interno tutte le dipendenze e le componenti necessarie all'esecuzione dell'applicazione, evitando così problemi di compatibilità
\item \textbf{Efficienza} - I container hanno la capacità di condividere il kernel della macchina ospitante evitando così di dover disporre di risorse hardware e software dedicate. Inoltre è sempre possibile aumentare e diminuire le risorse in modo istantaneo, in base alle proprie esigenze, in modo da non incorrere in sprechi o in carenze.
\item \textbf{Ottimizzazione dello spazio} - L'immagine dell'applicazione che viene eseguita all'interno del contenitore, incapsula solo ed unicamente le componenti e le informazioni necessarie alla sua esecuzione.
\end{itemize}