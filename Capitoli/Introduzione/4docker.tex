\section{Docker}
Docker è una tecnologia open–source per la costruzione, spostamento, distribuzione e rilascio di
applicazioni basate su container, distribuita sotto Licenza Apache Common 2.0, disponibile per
tutte le maggiori piattaforme in grado di eseguire container in ambienti Linux oppure Windows.
La struttura stessa dei container garantisce il loro isolamento, permettendo l’esecuzione multipla
su un singolo host, sfruttando direttamente il kernel della macchina senza la necessità di un server
hypervisor. In questo modo, i container sono il centro dello sviluppo dell’applicazione, la quale
viene costruita come insieme di container per esprimere le dipendenze, permettendone una facile
distribuzione e testing in ambienti diversi, fino alla messa in campo in produzione in un data
center locale, presso un cloud provider o una soluzione ibrida tra le precedenti.
Docker utilizza e crea differenti tipi di oggetti, come immagini e container. Un container costituisce 
un’istanza di un’immagine in esecuzione, con il quale è possibile interagire direttamente dalla
macchina host, e può essere visto come un’evoluzione di un processo, isolato dagli altri container
e dal sistema host.
Un’immagine, invece, costituisce un template di sola lettura, utilizzato per la creazione di un
container Docker. Solitamente, un’immagine viene costruita sulla base di un’altra immagine alla
quale vengono aggiunte le configurazioni necessarie e peculiari dell’applicazione. La definizione
di un’immagine avviene all’interno di un file, chiamato Dockerfile, nel quale vengono indicate
le istruzioni, ognuna delle quali aggiunge un layer all’interno dell’immagine finale, in modo tale
che, di volta in volta, vengano ricompilati solo gli strati che effettivamente sono stati modificati.