%!TEX root = ../../00main.tex

\section{Multi-cloud}
Il multi-cloud è una forma architetturale in cui si utilizzano risorse e servizi di Cloud Provider differenti convergendoli in una singola architettura eterogenea. 
Le motivazioni che muovono un'organizzazione ad adottare questo tipo di strategia sono varie:
\begin{itemize}
\item \textbf{Necessità differenti} - Ogni Cloud Provider propone la propria gamma di servizi che non sempre rispecchia a pieno le esigenze del cliente. La possibilità e quindi la libertà di poter scegliere e selezionare distintamente i singoli servizi concedono all'utente il vantaggio di poter coprire totalmente le proprie esigenze.
\item \textbf{Localizzazione} - Ogni Cloud Provider detiene le proprie infrastrutture dislocate in diverse aree geografiche, l'utente può decidere di utilizzarle in base alle proprie esigenze logistiche e legali.
\item \textbf{Prestazioni} - L'utilizzo di risorse provenienti da fornitori distinti, permette una distribuzione più ampia dei servizi dell'applicativo.
\end{itemize}