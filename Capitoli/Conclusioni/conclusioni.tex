In questo lavoro di tesi viene descritta un'importante funzionalità introdotta all'interno del linguaggio Fly \cite{ISISLab}, ovvero la possibilità di utilizzare cluster Kubernetes per l'esecuzione di applicativi Fly.\\

\section{Obiettivi raggiunti}
L'introduzione e l'integrazione di un ambiente Kubernetes all'interno di Fly ha permesso non solo di superare le limitazioni dettate dall'utilizzo di servizi Serverless ma ha permesso anche una maggiore libertà
di gestione delle risorse applicative.

\section{Sviluppi futuri}
Riguardo i possibili sviluppi futuri che possono interessare l'ambiente Kuberntes in Fly sono:

\begin{itemize}
    \item \textbf{Gestione dello storage} - La gestione dello storage in Kubernetes è estremamente potente e potrebbe portare grandi vantaggi
    \item \textbf{Bilanciamento del carico} - Kubernetes ha un proprio Load Balancer che se sfruttate le potenzialità aiuterebbe a gestire ancora meglio le risorse
\end{itemize}