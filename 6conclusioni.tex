In questo lavoro di tesi vengono descritte due importanti funzionalità introdotte all'interno del linguaggio Fly \cite{ISISLab}, ovvero la possibilità di utilizzare un ambiente di debug e il supporto alle sorgenti dati esterne.\\
Fly è un Domain-Specific Language che ha l’obiettivo di riconciliare il Cloud con il paradigma di calcolo ad alte prestazioni fornendo uno strumento potente, efficace ed efficiente che consenta di sfruttare il paradigma multi-cloud per lo sviluppo di applicazioni di Calcolo Scientifico. Un aspetto importante da considerare durante questo processo di sviluppo è la necessità di effettuare fasi di testing e di debug le quali risultano più complesse quando si lavora su Cloud. Il Cloud Computing, infatti, utilizza il sistema di costo pay-as-you-go secondo cui il prezzo viene calcolato in base all'effettivo utilizzo, in particolare in ambito FaaS ogni singola esecuzione di una funzione comporta un costo. Questo modello può rivelarsi un problema quando ci si approccia alle fasi di test di un'applicazione in cui sono necessarie numerose esecuzioni per verificare il suo corretto funzionamento, esecuzioni che seppur vanno in errore vengono calcolate nel costo complessivo. Oltre a ciò si aggiungono ulteriori problemi che vanno dai tempi di attesa dovuti ai caricamenti di dati o alla configurazione dei servizi, un minor controllo dell'ambiente fino alla necessità di una connessione di rete. Risulta quindi chiara la necessità di implementare all'interno del linguaggio Fly un ambiente che consenta di effettuare fasi di test e debug delle proprie applicazioni, simulando fedelmente l'esecuzione dell'ambiente Cloud ma allo stesso tempo evitandone i conseguenti costi. La soluzione sviluppata in questo lavoro sfrutta LocalStack \cite{LocalStack}, un framework che permette di eseguire su un container Docker \cite{docker} un ambiente del tutto simile al Cloud di Amazon Web Services, comprensivo di tutti i suoi servizi completamente funzionanti.\\
La seconda parte della tesi riguarda il supporto alle sorgenti dati esterne, in particolare la possibilità di utilizzare all'interno dei programmi Fly sia file dati strutturati che database relazionali, siano essi in locale o su Cloud. Rivolgendosi al mondo del Calcolo Scientifico è importante che Fly consenta di accedere a informazioni provenienti da fonti esterne, una funzionalità che finora non era presente nel linguaggio. I principali metodi utilizzati per la memorizzazione di dati riguardano file dati strutturati come CSV e fogli Excel insieme con i database relazionali, di cui il più utilizzato è sicuramente MySQL. Alla luce di ciò è fondamentale introdurre all'interno di Fly la possibilità di creare, leggere e scrivere file oltre che accedere e gestire database MySQL permettendo la creazione e l'esecuzione di query per l'ottenimento di dati. Volgendo l'attenzione verso il Cloud, per la gestione dei database i provider mettono a disposizione dei servizi basati sul modello Database as a Service, in particolare, per la gestione di database MySQL, Amazon Web Services offre il servizio Amazon RDS \cite{RDS} mentre Microsoft Azure mette a disposizione il servizio Azure for MySQL \cite{AzureMySQL}. Questi sono stati implementati all'interno di Fly utilizzando le soluzioni e gli strumenti più validi in base al contesto, sfruttando il grande vantaggio del modello DBaaS che consente di accedere ai database su Cloud allo stesso modo di come si farebbe con database locali.

\section{Obiettivi raggiunti}
Gli obiettivi posti sono stati raggiunti con successo per entrambe le funzionalità introdotte. \\
L'utilizzo di LocalStack ha permesso di introdurre un ambiente di debug che sia facile da utilizzare e che abbia pochi requisiti di utilizzo, permettendo di eseguire le applicazioni Fly senza interagire in alcun modo con l'ambiente su Cloud simulandone allo stesso tempo l'esecuzione in modo accurato. La sua struttura basata su Docker consente di avere il pieno controllo sull'ambiente, potendo di personalizzare il container per essere in linea con le necessità di Fly. L'implementazione è stata progettata in modo tale da consentire all'utente di utilizzare l'ambiente di debug allo stesso modo di come farebbe con qualsiasi altro ambiente di esecuzione, svincolandolo anche dalla necessità di avere un account AWS attivo.\\
L'introduzione del supporto alle sorgenti dati esterne è stato studiato in modo da raggiungere gli obiettivi implementativi posti in principio, che si concretizzano nella possibilità di utilizzare file dati strutturati e database MySQL. Una serie di entità e metodi specifici sono stati aggiunti al linguaggio Fly per permettere la lettura, la scrittura e la creazione di file dati strutturati, siano essi CSV, fogli Excel o semplici txt. Allo stesso modo l'utente può accedere e gestire ai propri database MySQL sia su Cloud, coi provider AWS o Azure, che in locale e ha la possibilità di creare ed eseguire query per la manipolazione dei dati utilizzando agevolmente i risultati ottenuti all'interno del programma.

\section{Sviluppi futuri}
Riguardo i possibili sviluppi futuri che possono interessare il linguaggio Fly due ambiti sono particolarmente interessanti e seguono il filone della gestione dei dati, ovvero l'implementazione del supporto a \textbf{database NoSQL} e alle \textbf{query distribuite}. Entrambi sono argomenti di grande interesse nella letteratura scientifica e stanno vivendo un grande sviluppo negli ultimi anni, soprattutto spinti dall'espansione del paradigma del Cloud Computing. I database NoSQL, infatti, permettono di sfruttare al meglio la scalabilità caratteristica del Cloud e offrono maggiore flessibilità nella gestione dei Big Data, mentre le query distribuite consentono di trarre il meglio dal paradigma del multi-cloud sia facilitando la gestione di dati replicati, o comunque presenti su diversi database, sia aumento le prestazioni e la sicurezza.

\subsection{Database NoSQL}
NoSQL è un movimento che promuove sistemi software dove la persistenza dei dati è caratterizzata dal fatto di non utilizzare il modello relazionale, di solito usato dai database tradizionali (RDBMS). Il termine NoSQL è stato usato per la prima volta nel 1998 per una base di dati relazionale open source che non usava un’interfaccia SQL. Il termine fu poi introdotto nel 2009 in un evento che trattava le basi di dati non relazionali, distribuite e che non offrono le tradizionali garanzie ACID e delle transazioni.\\
NoSQL è l’acronimo di \textit{“Not only SQL”} e viene utilizzato per indicare database che non si basano sul modello relazionale e che potrebbero non avere SQL come linguaggio di interrogazione. I database NoSQL sono appositamente realizzati per modelli di dati specifici e particolarmente adatti in applicazioni moderne che necessitano di schemi flessibili, alte prestazioni e alta scalabilità. Sono caratterizzati da una serie di proprietà:

\begin{itemize}
    \item \textbf{non seguono il modello relazione} - i dati sono memorizzati in vari formati e non solo in tabelle;
    \item \textbf{sono schemaless} - non esiste uno schema fisso predeterminato che i dati devono assolutamente rispettare, essi possono cambiare nel tempo;
    \item \textbf{non utilizzano il linguaggio SQL} - l’estrazione dei dati non utilizza le JOIN ma avviene attraverso semplici interfacce, inoltre permettono la manipolazione dei dati a basso livello;
    \item \textbf{nessun mapping oggetto-relazione o normalizzazione dei dati};
    \item \textbf{scalano in modo orizzontale} - i carichi di lavoro sono distribuiti su più macchine permettendo così di ottenere elevate prestazioni a costi ridotti.
\end{itemize}

I database NoSQL possono essere divisi in quattro categorie principali, ognuna delle quali ha proprie caratteristiche peculiari e, pertanto, un contesto più adatto di utilizzo.

\begin{itemize}
    \item \textbf{chiave-valore} - utilizzati quando non è possibile definire uno schema sui dati ed è necessario un accesso rapido alle singole informazioni possibile mediante l'utilizzo delle chiave per ottenere il valore. Sono usati principalmente per per memorizzare informazioni che non presentano correlazioni tra loro;
    \item \textbf{basati su colonne} - memorizzano i dati per colonne invece che per righe, permettendo la compressione dei dati e il versioning. Sono tipicamente utilizzati nell’indicizzazione di pagine web che cambiano spesso nel tempo e possiedono testo facilmente comprimibile;
    \item \textbf{basati su documenti} - caratterizzati da una struttura fondamentale detta \textit{document}, sfruttano il formato JSON, in cui un identificatore univoco è seguito da una serie di attributi. La loro natura gerarchica, semistrutturata e flessibile gli permette di evolversi in base alle esigenze. Sono utili in caso di dati che variano nel tempo e per mappare dati nel modello OOP;
    \item \textbf{basati su grafi} - memorizzano le informazioni nella forma di grafi con nodi interconnessi tra loro, di conseguenza sono particolarmente adatti per memorizzare dati fortemente correlati, fornendo alte prestazioni mediante l'uso di algoritmi di attraversamento. Sono tipicamente usati in contesti come social network, sistemi di raccomandazione o rilevamento di frodi.
\end{itemize}

I database NoSQL sono supportati da tutti i maggiori Cloud provider, in particolare Amazon Web Services mette a disposizione un insieme di servizi ognuno dei quali è specifico per una particolare famiglia di database \cite{AwsDb}. Al contrario Microsoft Azure fornisce il supporto al generico modello NoSQL tramite il servizio Azure Cosmos DB \cite{Cosmos}.

\subsection{Query distribuite}
Definiamo query distribuita un'interrogazione per l’acquisizione di dati memorizzati su diverse istanze di database, essa viene eseguita da un client su un solo server ma accede a dati memorizzati su più macchine, solitamente un sistema distribuito. L’utente che utilizza questo tipo di query è consapevole di attingere a più sorgenti dati ma il risultato che ottiene e la sintassi che utilizza sono analoghi a quelli di una classica interrogazione lanciata su un singolo database. \\
L’obiettivo all’interno di Fly è di permettere all’utente di utilizzare una normale query SQL per accedere a diverse sorgenti dati, siano esse locali o in cloud. La gestione delle interazioni coi diversi database e il raggruppamento del risultato verrà totalmente astratto, le informazioni richieste saranno fornite nella stessa forma prevista dalle query comuni. \\
L'attuale idea di implementazione prevede l'introduzione di una nuova entità che consenta di specificare diverse sorgenti dati, a partire da questa il compilatore provvederà alla creazione delle query singole che andranno in esecuzione su ogni database specificato nel campo \verb|source|. I vari risultati ottenuti verranno poi raggruppati in una singola tabella che l'utente potrà gestire normalmente.\\
Il Listato~\ref{lst:distributedQuery} mostra un esempio di query distribuita in Fly in cui viene dichiarata un'entità \verb|distributed| che permette di selezionare i dati presenti in tre diversi database. Il compilatore utilizza la stringa contenuta nel campo \verb|query| per generare tre diverse query che all'esecuzione saranno lanciate sui database specificati nei campi \verb|source1|, \verb|source2| e \verb|source3|.\\
\begin{lstlisting}[language=FLY,caption={Esempio di una Distributed Query in Fly.}, label={lst:distributedQuery}]
var queryStipendio = [type="distributed", query_type="value", query="SELECT nome, cognome, codiceFiscale, AVG(stipendioNetto) FROM db.anagrafica, dbAzure.anagrafica, dbAws.contabilita", source1=dbConnLocal, source2=dbConnAzure, source3=dbConnAWS]

var datiStipendio = queryStipendio.execute()
\end{lstlisting}