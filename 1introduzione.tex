%!TEX root = 00main.tex

%To-Do%
\section{Introduzione}
Lorem ipsum dolor sit amet, consectetur adipisicing elit, sed do eiusmod
tempor incididunt ut labore et dolore magna aliqua. Ut enim ad minim veniam,
quis nostrud exercitation ullamco laboris nisi ut aliquip ex ea commodo
consequat. Duis aute irure dolor in reprehenderit in voluptate velit esse
cillum dolore eu fugiat nulla pariatur. Excepteur sint occaecat cupidatat non
proident, sunt in culpa qui officia deserunt mollit anim id est laborum.

\section{Cloud Computing}
\subsection{Panoramica}
Il Cloud Computing è una particolare ed innovativa forma di erogazione di risorse da un fornitore ad un cliente attraverso la rete internet. Esso segue un paradigma architetturale di tipo distribuito ed i servizi offerti possono essere vari come l'archiviazione, la computazione oppure una semplice trasmissione dati.
Le risorse e le configurazioni a cui fa fede, non sono predefinite ma, con l'utilizzo di processi automatizzati, ogni
%Approssimativo perchè esiste anche la possibilità di crearsi la propria infrastruttura cloud%
 Cloud Provider è in grado di personalizzare e scalare le risorse opportune per quel determinato servizio a quel determinato cliente. Una volta che quest'ultimo termina l'utilizzo delle risorse assegnate, queste tornano nuovamente disponibili, pronte per una nuova assegnazione.

\subsection{Vantaggi}
E' possibile individuare i maggiori vantaggi che motivano le aziende e gli sviluppatori a utilizzare il Cloud Computing per i propri applicativi o per le proprie infrastrutture.
%In realtà non è del tutto corretto perchè l'azienda può scegliere di costruirsi la propria infrastruttura su un Cloud locale%
\subsubsection{Costi}
E' chiaro che in quest'ottica, il modello economico Pay-as-you-go è diventato parte integrante della strategia di fruizione dei servizi, in quanto porta innumerevoli vantaggi ai fornitori ma sopratutto ai clienti stessi, tra cui:
    \begin{itemize}
    \item \textit{Ottimizzazione del capitale} - la libertà di pagare solo ciò di cui si ha realmente bisogno, l'eliminazione dei costi di investimento e dei costi di mantenimento ed il miglioramento dell'efficienza energetica permettono un notevole abbattimento dei costi ed un'ottimizzazione del capitale.
    \item \textit{Flessibilità} - I clienti, in ogni momento ed in maniera quasi istantanea, hanno la massima libertà di poter annullare, sostituire o fermare qualsiasi servizio in corso.
    \item \textit{Scalabilità} - Il sistema adottato, accompagna senza problemi la crescita di piccole infrastrutture. In effetti, per una Start-up agli inizi, è pratico dotarsi di uno strumento adeguato in termini di prezzi e funzionalità, per poi farlo crescere in parallelo con l'evoluzione e le ambizioni dell'azienda.
    \end{itemize}

\subsubsection{Performance e scalabilità}
La tecnologia alla base del Cloud Computing è la virtualizzazione la quale, date le sue potenzialità, permette di ridimensionare agilmente le risorse assegnate. La potenza computazionale, la larghezza di banda e lo spazio di archiviazione sono esempi di risorse che possono essere scalate in base al carico di richieste, andando quindi ad allocarne una quantità maggiore in caso di maggiore attività, una quantità minore in caso di minore attività e a deallocare quelle inutilizzate in caso d'inattività prolungata.

\subsection{Modelli di servizio}
Il Cloud Computing è un mondo in continua crescita ed evoluzione, nuovi prodotti e servizi cloud arrivano quasi ogni giorno, spinti dalle costanti innovazioni tecnologiche.
Nonostante la maturità del mercato, molte organizzazioni non sono ancora a conoscenza dei servizi e dei modelli d'implementazione che i diversi Cloud Provider forniscono.
I tre modelli principali a oggi maggiormente diffusi sono così definiti:

\subsubsection{Software as a Service (Saas)}
Software as a Service (SaaS) è il modello di servizio cloud che fornisce l'accesso a un prodotto software completo, eseguito e gestito dal fornitore del servizio.
In questo particolare modello, il fornitore ha la piena responsabilità sull'intero ciclo di vita del software e sulla manutenzione dell'infrastruttura sottostante. Questo consente al cliente di concentrarsi esclusivamente su come utilizzare al meglio le specifiche e le funzionalità del software offerto.

\subsubsection{Platform as a Service (Saas)}
Platform as a Service (PaaS) è il modello usato più comunemente per lo sviluppo di applicazioni. Il fornitore del servizio fornisce l'accesso alle proprie risorse infrastrutturali come: basi di dati, sistemi operativi e server, senza la complessità di gestione che ne deriva. Questo permette al cliente di dedicare le proprie risorse all'ottimizzazione dell'applicativo invece che all'installazione e alla configurazione dell'infrastruttura.

\subsubsection{Infrastructure as a Service (Saas)}
Infrastructure as a Service (IaaS) è il modello che permette al cliente d'implementare la propria infrastruttura cloud. Il Cloud Provider fornisce l'accesso on-demand delle proprie risorse (sia fisiche che virtuali) volte alla computazione, archiviazione e alla rete.
Il cliente ottiene così l'enorme vantaggio di poter distribuire e gestire i propri applicativi avendo la sicurezza di avere risorse sempre disponibili, affidabili e scalabili.

\subsection{Modelli di distribuzione}
Ogni modello d'implementazione del cloud ha una propria configurazione unica con una gamma di requisiti diversi e vantaggi associati.

\subsubsection{Public Cloud}
Il Public Cloud è un modello di distribuzione in cui i servizi cloud sono di proprietà di fornitori esterni. Scegliere questa metodologia garantisce una maggiore agilità operativa e una scalabilità pressoché illimitata, perché sfrutta le funzionalità e le risorse di grandi fornitori come Google, Microsoft e Amazon.
Trattandosi inoltre di ambienti gestiti, il cliente si solleva dalle responsabilità di manutenzione e controllo delle risorse utilizzate.

\subsubsection{Private Cloud}
Nel modello di distribuzione Private Cloud si sceglie di sviluppare, mantenere e gestire la propria infrastruttura cloud e di fornire l'accesso solo alla propria rete interna. Con il Private Cloud le organizzazioni mantengono un controllo completo dell'infrastruttura. Questo si traduce in un controllo completo delle proprie risorse e ad una maggior libertà di personalizzazione dei servizi, che sono consumati all'interno del cloud.

\subsubsection{Hybrid Cloud}
L’Hybrid Cloud è un modello di distribuzione ibrido in quanto 
Si sceglie di unire i vantaggi del cloud pubblico e di quello privato.
Ad esempio, si possono utilizzare le risorse del Public Cloud per attività di computazione, e tenere al sicuro i dati sensibili e le applicazioni critiche nel Private Cloud. Oppure, di fronte a picchi improvvisi e temporanei della domanda di risorse, le organizzazioni possono scegliere di spostare i carichi di lavoro dal Private Cloud al Public Cloud.
Utilizzato in questo modo, l’Hybrid Cloud consente di ottenere il meglio da entrambe le infrastrutture.